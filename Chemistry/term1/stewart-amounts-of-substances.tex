\section{Amounts of substances}
\subsection{The mole}
The mole, or Avagadro's constant, is the number of carbon-12 atoms in 12g of pure carbon-12. It is roughly $6.02\times 10^{23}$. The relative atomic mass is the weight of 1 mole of atoms of an element, if those atoms follow the normal isotopic distribution of the element in question.

This could also be said as $\text{amount } n = \frac{\text{mass } m}{\text{molar mass } M}$.

\subsection{Formulae}
The empirical formula is the simplest whole-number ratio of atoms of each element in a compount.

The molecular formula is a multiple of the empirical formula, and is the number of atoms which are actually in one molecule of the substance in question.

\subsection{Finding formulae}
When given the proportion of a substance's mass that is made up of specific elements, it is possible to find the substance's empirical formula. The way to do this is to divide the masses for each element by the elements' relative atomic masses to get the number of moles, and then each of these by the smallest one to give a molar ratio, which can be rounded slightly or multiplied up to give a whole number ratio. If given the molecular mass, it's also possible to find the molecular formula by dividing the molecular mass by the relative formula mass for the empirical formula, and then multiplying the empirical formula by the answer.
