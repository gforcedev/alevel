\section{Ionisation Energy}
Ionisation energy can be defined as the amount of energy required to remove 1 mole of electrons from 1 mole of gaseous particles, leaving 1 mole of gaseous ions with a more positive charge. For example:
\begin{equation}
	\ce{O^{2+}(g) -> O^{3+} + e-}
\end{equation}

\subsection{Factors affecting ionisation energy}
\subsubsection{Nuclear charge}
The overall charge of the nucleus, affected by how many protons the nucleus has. Increases down groups, and to the right of periods.
\subsubsection{Atomic radius}
How far the electrons are from the nucleus. Affected by the number of the shell containing the electron in question, and also by how many shells there are in relation to the nuclear charge (fewer shells will be held closer to a nucleus of the same positive charge, and more charge will hold the same number of shells closer).
\subsubsection{Shielding}
The amount of shielding happening because of electrons between the nucleus and the electron in question. Affected by how many shells are between the nucleus and the electron in question, as electrons on the same shell don't contribute to the shielding effect.

\subsection{Trends in first ionisation energy down groups}
Moving down groups, the nuclear charge increases. However, this increase in nuclear charge is outweighed by the increase in shielding, and increase in atomic radius, meaning that the first ionisation energy will decrease moving down groups
\subsection{Trends in first ionisation energy along periods}
In periods, as we move to the right then the nuclear charge increases, and the atomic radius decreases because the charge increased but the number of shells didn't. There is also no change to shielding because we didn't add any more shells. This means that, in general, the first ionisation energy will increase moving to the right of periods.
\subsubsection{Anomalies regarding the general trend along periods}
There are two exceptions to this general trend. That is to say, there are two points along a period will decrease rather than increase the first ionisation energy They occur at group 3 and group 6.
\\
At group 3, we go from removing the second electron from the s subshell to removing the first electron from the p subshell. The p subshell requires less energy to remove an electron from, and so the first ionisation energy is lower. For example, boron 1s$^2$ 2s$^2$ 2p$^1$ has a lower first ionisation energy than beryllium 1s$^2$ 2s$^2$, even though it is further to the right of the period.
\\
At group 6, we go from removing the third electron from the p subshell, to removing that same p subshell's fourth electron. This causes an anomaly because of Hund's rule, which states that each orbital will contain one electron before any contain two. This means that we are now removing an electron from an orbital containing two electrons, and its paired electron will repel it slightly, lowering the energy required to remove it and so again lowering the first ionisation energy. For example, oxygen 1s$^2$ 2s$^2$ 2p$^4$ has a lower first ionisation energy than nitrogen 1$^2$ 2s$^2$ 2p$^3$, even though it is further to the right of the period.
