\section{Covalent Bonding}
Covalent bonds are formed between two non-metals, and based on shared pairs of electrons. That is to say, pairs of electrons are shared between atoms so that they can both gain electrons and reach noble gas electron configuration.

\subsection{How covalent bonds form}
Electrons are shared by the atoms moving together so that orbitals containing the paris of electrons overlap. The bond can be described as the attraction between a positively charged nucleus, and a shared pair of electrons.

\subsection{Representing covalent bonds}
\subsubsection{Dot and cross diagrams}
Dot and cross diagrams can be used to help us understand covalent bonds.
\begin{figure}[ht]
    \centering
    \incfig{dot-and-cross-diagram-for-h2}
	\caption{Dot and cross diagram for $\ce{H2}$}
    \label{fig:dot-and-cross-diagram-for-h2}
\end{figure}
\subsubsection{When drawing molecules}
Covalent bonds can be represented by drawing solid lines between the atoms' symbols. The number of lines represents how many pairs of electrons are shared. Each line is one shared pair of electrons. For example, ethene can be drawn like this:\\
\chemfig{C (-[3]H) (-[5]H) = C (-[1]H) (-[7]H)}

\subsection{Different kinds of electron pair}
A pair of electrons involved in a covalent bond is called a \textbf{bonding pair}. A pair of electrons not involved in a covalent bond (but in a covalently bonded molecule) is called a \textbf{lone pair}, and may be represented by a double dot when drawing molecules. For example a diagram of, $\ce{NH3}$ might be drawn something like this:
\chemfig{\lewis{2:,N} (-[4]H) (-[6]H) (-[8]H)}

\subsection{Properties of covalently bonded substances}
Covalent substances don't conduct electricity because they have no mobile ions or electrons.
\\
Covalent substances tend to be more soluble in organic solvents than in water; some are hydrolysed.
\\
Boiling points in covalent substances are low because intermolecular forces are weak, and so little energy is required to separate molecules from each other. Boiling points increase as the molecules get a larger surface area. An example would be the alkanes getting higher boiling points as chains get longer.

\subsection{Not needing NGEC}
As long as the middle atom maximises the number of covalent bonds it has, the compount will be stable. For example, in $\ce{BF3}$, boron only has 6 electrons in its outer shell, but because it has 3 covalent bonds then the molecule is stable.
\subsection{Expanding the octet (going beyond 8 electrons)}
For elements further down the periodic table, more than 4 covalent bonds can be formed. This is because the n$=3$ shell can hold 18 electrons, and so more are available for bonding, for example in phosphorus pentachloride ($\ce{PCl5}$).
\subsubsection{What's really going on (NOT IN THE SPEC)}
The phosphorus has the configuration [Ne]3s$^2$3p$^3$, with the 3s electrons being paired and the 3p electrons being unpaired (because of hund's rule). What happens is that one of the 3s electrons moves to the first 3d orbital, meaning that there are 5 electrons in the outer shell, all unpaired. This means that there can be 5 covalent bonds formed. This is known as hybridisation. In this example, the hybridisation has created the sp$^3$d hybrid orbitals (5 hybrid orbitals in total).


\subsection{Dative covalent bonds (or coordinate bonds)}
Dative covalent bonds are slightly different from covalent bonds in their formation. The shared electrons making up the bonding pair are both from one donor atom, rather than one being from each of the two that are sharing. For example, $\ce{NH4+}$ ions are formed by the lone pair from an $\ce{NH3}$ molecule being donated to an $\ce{H^+}$ ion. This can be displayed as an arrow, like this:\\
\[\chemfig{N (-[0,,,,->]H^+) (-[2]H) (-[4]H) (-[6]H)}\]$^+$
