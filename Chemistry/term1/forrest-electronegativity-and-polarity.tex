\section{Electronegativity and polarity}
Electronegativity is a measure of the tendency of an atom to attract a bonding pair of electrons in a covalent bond. It is affected by the same factors affecting ionisation energy: atomic radius, nuclear charge, and electron shielding.

\subsection{Polar covalent bonds}
When atoms have different levels of electronegativity, the more electronegative atom will pull the bonding electrons towards it. For example, in \ce{HCl}, chlorine is more electronegative than hydrogen, and so the elecrons are pulled towards it. We show this with a lowercase delta. This change in charge is called a permanent dipole.
\begin{center}
	\chemfig{\chemabove[3pt]{H}{\delta +}-\chemabove[3pt]{Cl}{\delta -}}
\end{center}

\subsection{Polarity of molecules}
If a molecule has permanent dipoles and is non-symmetrical, then the molecule is considered polar. For example, with water, then it has a permanent dipole across both of the bonds, and then are non-symmetrical, as they'd have to be vertically opposite to cancel out but the molecule has an angular shape.
\begin{center}
	\chemfig{\chemabove[3pt]{O}{\delta +}(-[-1]\chemabove[3pt]{H}{\delta -})(-[-3]\chemabove[3pt]{H}{\delta -})}
\end{center}

A different example would be tetrachloromethane, \ce{CCl4}. It has four permanent dipoles, but they cancel in the tetrahedral shape leaving a non-polar molecule.
\begin{center}
	\chemfig{
		\chemabove[3pt]{C}{\delta +}
		(-[3]\chemabove[3pt]{Cl}{\delta -})
		(-[5]\chemabove[3pt]{Cl}{\delta -})
		(<[1]\chemabove[3pt]{Cl}{\delta -})
		(<:[-1]\chemabove[3pt]{Cl}{\delta -})
	}
\end{center}

\subsection{The Pauling scale}
The Pauling scale is a scale for measuring elecronegativity. It was invented by Linus Pauling. This is important because the greater the difference in electronegativity, the greater the permanent dipole formed.

\subsection{Trends in electronegativity}
\subsubsection{To the right of periods}
Electronegativity increases to the right of periods. This is because the shielding stays the same, while the nuclear charge increases and the atomic radius decreases. This means that there's more charge pulling the pair towards the nucleus, and this is compounded by the fact that the pair will be closer to the nucleus in the first place (and so more susceptible to its attraction).
\subsection{Down groups}
Electronegativity decreases down groups. This is because the shielding increases, and the atomic radius increases. The nuclear charge increases, but not enough to counteract the increase in both shielding and atomic radius, making the overall electronegativity decrease down the group.

