\section{Electron Configuration}
Atom's electrons go inside different shells. These shells are split into different subshells: s, p, d, and f. Each shell contains one more subshell than the one below it. That is to say, the first shell contains only the s subshell, the second shell contains the s and p subshells, the third shell contains s, p, and d, and so on. The subshells contain different numbers of orbitals
\subsection{Electron Orbitals}
An orbital can be defined as a region of space in an atom that can hold up to 2 electrons with opposite spins. The different subshells contain different numbers of orbitals.
\subsubsection{Orbitals in the s subshell}
The s subshell contains one spherical orbital. This means that the s subshell can hold up to 2 electrons
\subsubsection{Orbitals in the p subshell}
The p subshell contains three orbitals, that are lobe shaped (similar to a 3-dimensional number 8). There is one on each of the $x$, $y$, and $z$ axes. This means the p subshell can hold up to 2 electrons
\subsubsection{Orbitals in further subshells}
For A-level, knowledge of the shapes of d and f orbitals is not required. What is required is knowing that the number of orbitals keeps going up by 2. So the d subshell has 5 orbitals, and can hold up to 10 electrons. The f subshell has 7 orbitals, and can hold up to 14 electrons.

\subsection{The size of the shells}
Based on this knowledge, we can work out the sizes of all of the shells.
\begin{center}
\begin{tabular}{|c|c|c|c|c|}
\hline
Shell     & 1st Shell & 2nd Shell & 3rd Shell & 4th Shell   \\
Subshells & 1s        & 2s 2p     & 3s 3p 3d  & 4s 4p 4d 4f \\
Electrons & $2$       & $2+6$     & $2+6+10$  & $2+6+10+14$ \\
Total     & $2$       & 8         & 18        & 32          \\
\hline
\end{tabular}
\end{center}

\subsection{Writing electron configurations}
\subsubsection{Full electron configurations}
When writing out the electron configuration of an atom or ion, we write out each subshell, with the amount of electrons it contains in superscript. For example, the electron configuration of a sodium atom could be written as 1s$^2$ 2s$^2$ 2p$^6$ 3s$^1$
\subsubsection{Short form electron configurations}
If an electron configuration is extremely long, you can write it in short form. This means writing the nearest preceeding noble gas in square brackets, and then writing out just the end of the configuration. Using this, a sodium atom's electron configuration can be written as [Ne] 3s$^1$.

\subsection{Filling orbitals}
\subsubsection{Hund's rule}
Hund's rule states that each orbital in a subshell is singularly occupied before any are doubly occupied. For example, if there are 3 electrons in a p subshell, each orbital will contain just 1 electron, rather than one orbital having 2 and another 0. In addition, Hund's rule states that each of the electrons in singly occupied orbitals will have the same spin.
\subsubsection{The 4s subshell}
The 4s subshell has a lower energy than the 3d subshell, and so fills \textbf{before} the 3d subshell. Googling for ``Electron Configuration Table'' yields images which are helpful for understanding how this affects the layout of the periodic table.
