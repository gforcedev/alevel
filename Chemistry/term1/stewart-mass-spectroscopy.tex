\section{Mass spectroscopy}
\subsection{Masses}
\subsubsection{Relative isotopic mass}
The relative isotopic mass of an isotope is the amount of grams one mole of that isotope would weigh, relative to a twelfth of the mass of carbon-12.
\subsubsection{Relative atomic mass}
The relative atomic mass of an element is the average mass of each atom of that element. We can find this by getting the relative isotopic masses of each naturally occuring isotope of the element, as well as their percentage abundances. We then multiply the isotopic masses by the corresponding percentage abundances and add our answers together to get the relative atomic mass.

\subsection{How mass spectroscopy works}
Mass spectroscopes work by ionising the atoms inside into positive ions. They then accelerate these ions, and as ions of different masses move more slowly, each isotope's ions are separated. The ions show up on the mass spectrometer as mass-to-charge, or $M/z$ ratios. As the ions are normally $1^+$ ions, and the mass of one electron is negligible, then the $M/z$ value is equal to the relative isotopic mass.

\subsection{Analysing mass spectroscopy graphs}
\begin{figure}[ht]
    \centering
    \incfig{an-example-of-a-mass-spectroscopy-graph}
    \caption{An example of a mass spectroscopy graph}
    \label{fig:an-example-of-a-mass-spectroscopy-graph}
\end{figure}
To work out the percentage abundance, divide the peak height by the total height of all peaks, and multiply by 100. Then multiply the percentage abundances by the mass number and sum these values to get the relative atomic mass of the element in question.
