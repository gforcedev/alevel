\section{Ionic bonding}
Ionic bonds form between particles with relatively low ionisation energies, usually between metals and non-metals. If ionisation energies are high, covalent bonding will be favoured.
\subsection{How ionic bonds form}
Ionic bonds form when a metal loses electrons in order to achieve noble gas electron configuration. For example, sodium could lose one electron to form an $\ce{Na^+}$ cation. This electron is then gained by the non-metal, for example chlorine, in order to form a $\ce{Cl^-}$ anion. Electrostatic attraction between the two particles is what makes this into a bond, as the two particles are now chemically held together.

\begin{IEEEeqnarray}{C}
	\ce{Na -> Na^+ + e^-}
	\nonumber\\
	\ce{Cl + e^- -> Cl^-}
\end{IEEEeqnarray}
\begin{figure}[ht]
    \centering
    \incfig{an-ionic-bond}
	\caption{How ionic bonds are formed}
    \label{fig:an-ionic-bond}
\end{figure}

\subsection{Giant ionic lattices}
Ionic substances (as solids) are often held in giant ionic lattices. These are regular 3-dimensional lattices held together by electrostatic attraction. For the example of Sodium Chloride, each $\ce{Na^+}$ is surrounded by 6 $\ce{Cl^-}$ ions (one on each side on all 3 axes). Each $\ce{Cl^-}$ is surrounded in the same way by $\ce{Na^+}$ ions.

\subsection{Properties of ionic substances}
Melting points of ionic substances are high because of the large amount of energy required to overcome the strong electrostatic attraction between ions.
\\
Ionic substances are brittle because any dislocation causes layers to misalign, meaning that the ions will repel rather than attract, and splitting the substance.
\\
Ionic substances don't conduct electricity when solid, but do conduct when molten or when in aqueous solution. This is because the ions are free to move.
\\
Ionic substances are insoluble in non-polar solvents, but soluble in polar ones (like water). This is because polar substances are able to break down the lattice and surround each ion in solution.
