\section{Reactivity trends in group 2}
\subsection{Reactivity and ionisation energy}
Going down group 2, atomic radius increases, and so does shielding. Nuclear attraction also increases, but not enough to counteract these first two increases, meaning that ionisation energies and reactivity decreases down the group.

\subsection{Reactions of group 2 elements}
Group 2 elements are reducing agents. This means that in reactions they lose electrons (get oxidised) so that the other reactant can be reduced.

\subsubsection{Reactions with oxygen}
Group 2 metals react vigorously with oxygen producing group 2 oxides. For example:
\begin{equation}
	\ce{2Ca{(s)} + O2_{(g)} -> 2CaO_{(s)}}
\end{equation}
In this reaction, the calcium is oxidised from 0 to 2+, while the oxygen is reduced from 0 to 2-.

\subsubsection{Reactions with water}
Group 2 metals react with water to form a group 2 hydroxide. For example:
\begin{equation}
	\ce{Ca_{(s)} + 2H2O_{(l)} -> Ca(OH)2_{(g)}}
\end{equation}
In this reaction, the calcium is oxidised from 0 to 2+, while the hydrogen is reduced from 1+ to 0.

\subsubsection{Reactions with acids}
Group 2 metals react with dilute acids to form a salt and hydrogen. For example:
\begin{equation}
	\ce{Ca_{(s)} + 2HCl_{(eq)} -> CaCl2_{(aq)} + H2_{(g)}}
\end{equation}
In this reaction, the calcium is oxidised from 0 to 2+, while the hydrogen is reduced from 1+ to 0 (chlorine is a spectator ion in this case).

\subsection{Solubility of group 2 hydroxides}
The solubility of group 2 hydroxides increases down the group. This means that more particles can be dissolved into the same amount of solvent, also effectively increasing the alkilinity of saturated group 2 hydroxide solutions.

\subsection{Reactions of group 2 compounds}
\subsubsection{Group 2 oxides}
Group 2 oxides react with acids to produce group 2 salts and water. For example:
\begin{equation}
	\ce{MO + 2HCl -> 2MCl2 + H2O}
\end{equation}
The salt formed will dissolve in water.

\subsubsection{Group 2 carbonates}
Group 2 carbonates also react with acids, producing group 2 salts, water, and carbon dioxide. For example:
\begin{equation}
	\ce{MCO3 + 2HNO3 -> M(NO3)2 + H2O + CO2}
\end{equation}

Group 2 carbonates also decompose when heated into group 2 oxides and carbon dioxide:
\begin{equation}
	\ce{MCO3 -> MO + CO2}
\end{equation}
This reaction, when calcium is involved, is part of the limestone cycle.

\subsubsection{Industrial uses of group 2 compounds}
These reactions mean that both group 2 oxides and carbonates are used in neutralising acids. This is done industrially in overly acidic soils so crops can grow, and also in indigestion remedies for neutralising stomach acid.
