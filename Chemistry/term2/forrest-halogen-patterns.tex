\section{The Halogens}
\subsection{Colours and states}
Fluorine is a pale yellow gas at room temperature.
\\
Chlorine is a pale green gas at room temperature.
\\
Bromine is an orange liquid at room temperature.
\\
Iodine is a purple or shiny black solid at room temperature. Its crystals also sublime when heated.

\subsection{Trends down the group}
Going down group 7, the electronegativity decreases due to the usual factors of an increasing nuclear charge being counteracted by the increase in shielding and atomic radius. The reactivity decreases downward for these reasons as well, as it becomes harder for the nucleus to attract the extra electron required to achieve noble gas configuration.

\subsection{Displacement reactions}
More reactive halogens will displace less reactive ones in compounds. For example:
\begin{equation}
	\ce{Cl2 + 2KBr -> 2KCl + Br2}
\end{equation}
In this case there would be a colour change from pale green to red.

\subsection{In redox reactions}
Halogens are oxidising agents. This is because they accept electrons, being reduced from 0 to 1-, and thereby allowing other substances to be oxidised. For example:
\begin{equation}
	\ce{3Cl2 + 2Fe -> 2FeCl}
\end{equation}
In this case then the chlorines were reduced from 0 to 1-, which allowed the irons to be oxidised from 0 to 3+.

\subsubsection{Disproportionation reactions}
Disproportionation reactions happen when some atoms in a compound are reduce while others are oxidised. For example:
\begin{equation}
	\ce{Cl2 + H20 -> HCl + HOCl}
\end{equation}
In this reaction, one of the chlorine atoms was reduced from 0 in $\ce{Cl2}$ to -1 in $\ce{HCl}$, while the other was oxidised from 0 in $\ce{Cl2}$ to +1 in $\ce{HOCl}$. This is what makes it a disproportionation reaction.

\subsection{Identifying the halogens}
Halogens form solutions of different colours in water and cyclohexane:
\begin{table}[ht]
	\begin{tabular}{lll}
		Halogen  & Colour in water & Colour in cyclohexane \\
		Chlorine & Pale green      & Pale green            \\
		Bromine  & Orange          & Orange                \\
		Iodine   & Brown           & Violet               
	\end{tabular}
\end{table}

\subsection{Identifying halide ions}
To identify halide ions, add aqueous silver nitrate. Different coloured silver halide precipitates will form. Silver chloride is white, silver bromide is cream, and silver iodide is yellow. If you have samples of all 3 it should be easy to see by comparison, but a second test is to add ammonia to this sample. Silver chloride will completely dissolve, silver bromide will partially disolve, and silver iodide will not dissolve at all.
