\section{Acids}
The acids that are required knowledge for A-level are $\ce{HCl}$, $\ce{H2SO4}$, $\ce{HNO3}$, $\ce{CH3OOH}$, $\ce{NAOH}$, $\ce{KOH}$ and $\ce{NH3}$.

\subsection{Strong vs.\ weak acids}
Strong acids dissociate fully in aqueous solution, while weak acids only partially dissociate. One effect of this is that strong acids are more conductive than weak acids because they have a higher ion concentration. Similarly, sulphuric acid is more conductive that hydrochloric acid because of its extra ions when dissociated.

\subsection{Acid theories}
\subsubsection{Arrhenius theory}
In Arrhenius theory, acids produce hydrogens ions in solution, and bases produce hydroxixe ions in solution. Neutralisation then happens when these ions react producing water.

\subsubsection{Brønstead-Lowry theory}
In Brønstead-Lowry theory, an acid is a proton donor, and a base is a proton acceptor. Hydroxide ions are bases as they accept protons, forming water. In solution, acids produce $\ce{H+}$ ions which react with water to give hydronium (also called hydroxonium), $\ce{H3O+}$.
\begin{equation}
	\ce{H2O + HCl -> H3O+ + Cl-}
\end{equation}
Monoprotic, diprotic and triprotic acids give up one, two, and three protons respectively.

Brønstead-Lowry theory requires acids to be dissolved in water, but allows for soluble and insoluble bases.

\subsection{Reactions of acids}
\begin{IEEEeqnarray}{sCu}
	Metal + acid & $\ce{->}$ & salt + hydrogen
	\\
	Metal oxide + acid & $\ce{->}$ & salt + water
\end{IEEEeqnarray}
These are both neutralisation reactions, because their ionic equations cancel to $\ce{H+}$ and $\ce{OH-}$ ions combining to make water:
\begin{IEEEeqnarray}{rCl}
	\ce{NaOH_{(aq)} + HCl_{(aq)}} & \ce{->} & \ce{NaCl_{(aq)} + H2O_{(l)}}
	\nonumber\\
	\ce{Na^+_{(aq)} + OH^-_{(aq)} + H^+_{(aq)} + Cl^-_{(aq)}} & \ce{->} & \ce{Na^+_{(aq)} + Cl^-_{(aq)} + H2O_{(l)}}
	\nonumber\\
	\ce{H^+_{(aq)} + OH^-_{(aq)}} & \ce{->} & \ce{H2O_{(l)}}	
\end{IEEEeqnarray}

Neutralisation actions are exothermic, and stronger acids give a higher enthalpy change.

\subsection{Carbonates and acids}
Acid + carbonate $\ce{->}$ salt + water + carbon dioxide.

This is useful as an excess of the base can be added without turning the solution alkaline.
