\section{Alkenes}
\subsection{Stereoisomerism}
Stereoisomers are two molecules with the same molecular formulae, but that take up different shapes in space. In alkenes, it happens because of the limited rotation around the carbon-carbon double bond because of the $\pi$-bond's electron density above and below the $\sigma$-bond.

\subsection{Cahn-Ingold-Prelog}
Cahn-Ingold-Prelog is a system for assigning priority to different groups bonded to carbon atoms. The atom with the higher atomic number recieves higher priority. For groups, then the priority is assigned based on the first point of difference between the two groups.

\subsection{E-Z isomerism}
E-Z isomerism happens when there's a C=C double-bond, and different groups attached to each carbon.
\subsubsection{E isomerism}
E stands for the German \textit{entgegen}, which means against. E-isomers are isomers in which the higher priority groups bonded to the carbons are placed diagonally.
\subsubsection{Z isomerism}
Z stands for the German \textit{zusammen}, which means together. Z-isomers are isomers in which the higher priority  grouops bonded to the carbons are both above or both below the bond.

\subsection{Cis-trans isomerism}
Cis-trans isomerism is a special case of E-Z isomerism where each of the carbons has one hydrogen bonded. The cis isomer is the Z isomer (both hydrogens on the same side), and the trans isomer is the E isomer (hydrogens on different sides).
