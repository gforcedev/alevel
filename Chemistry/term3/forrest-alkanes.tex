\section{Alkanes}
Alkanes are saturated hydrocarbons. This means that every carbon in the chain has the maximum amount of possible hydrogens bonded. The carbons are joined to the other atoms by a single covalent bond ($\sigma$ bond). These are overlapping orbitals, each containing one electron and therefore two are shared between each atom. The bond angles are 109.5\textdegree.

Alkanes have the general formula C$_n$H$_{2n+2}$.

The shapes of alkanes are not fixed, as the $\sigma$ bonds allow rotation, acting as axes.

The boiling points of alkanes increase with chain length, because there is more surface contact for London forces to act. Branching slightly decreases boiling point because it reduces surface contact. For example, 2-metyl-propanol has a higher boiling point than propanol, but a slightly lower boiling point than butanol.

\subsection{Combustion of alkanes}
Alkanes combust in the presence of oxygen. There are three combustion equations for each alkane. Complete combustion forms carbon dioxide and water, while incomplete combustion forms either carbon monoxide and water or carbon (soot) and water:
\begin{IEEEeqnarray}{rCl}
	\ce{CH4 + O2} & \ce{->} & \ce{CO2 + 2H2O}
	\nonumber\\
	\ce{CH4 + 1\frac{1}{2}O2} & \ce{->} & \ce{CO + 2H2O}
	\nonumber\\
	\ce{CH4 + O2} & \ce{->} & \ce{C + 2H2O}
\end{IEEEeqnarray}

\subsection{Free radical substitution for bromination of alkanes}
When an alkane and bromine are combined under UV light, the $\ce{Br2}$ bonds can undergo homolytic fission creating bromine radicals. These radicals begin the stages of free radical substitution to brominate the alkane. This happens in three steps: initiation, propogation, and termination.

\subsubsection{Initiation}
The initiation step is when the bromine bond undergoes homolytic fission.
\begin{equation}
	\ce{Br2 -> 2 \bullet Br}
\end{equation}

\subsubsection{Propogation}
There are two propogation steps: When the bromine radical pulls a hydrogen away from the alkane forming an alkyl radical and hydrogen bromide, and then when the alkyl radical makes another bromine molecule undergo homolytic fission, brominating the alkyl radical and forming a bromine radical.
\begin{IEEEeqnarray}{rCl}
	\ce{CH3CH3 + \bullet Br} & \ce{->} & \ce{\bullet CH2CH3 + HBr}
	\nonumber\\
	\ce{\bullet CH2CH3 + Br2} & \ce{->} & \ce{CH2BrCH3 + \bullet Br}
\end{IEEEeqnarray}
The key part about propogation steps is they start with a molecule and a radical, and form another molecule and a radical allowing the reaction to keep propogating.

\subsubsection{Termination}
The termination steps happen when two radicals collide and form a bond. Examples could be two bromine radicals combining, or two alkyl radicals combining, or some combination thereof.
\begin{IEEEeqnarray}{rCl}
	\ce{2 \bullet Br} & \ce{->} & \ce{Br2}
	\nonumber\\
	\ce{2 \bullet CH2CH3} & \ce{->} & \ce{CH3CH2CH2CH3}
	\nonumber\\
	\ce{\bullet CH2CH3 + \bullet Br} & \ce{->} & \ce{CH2BrCH3}
\end{IEEEeqnarray}

\subsection{Limitations of radical substitution}
With radical substitution, there's no way of controlling whether more substitution will take place. For example, you may want $\ce{CH2Br}$, but there's no way to control more propogation reactions ending up with $\ce{CHBr2}$. Another problem is that there's no way of controlling whether longer alkanes get produced in termination, and then substituted onto. Finally, substitution may happen onto different parts of the chain.
