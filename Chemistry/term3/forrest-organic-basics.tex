\section{Organic chemistry basics}
\section{Classifications of hydrocarbons}
At their most basic, there are three classifications of organic molecules:
\begin{itemize}
	\item Aliphatic hydrocarbons are arranged with the carbon atoms in chains.
	\item Alicyclic hydrocarbons have their carbon atoms in rings.
	\item Aromatic hydrocarbons contain a benzene ring.
\end{itemize}
\subsection{Organic molecule nomenclature}
Organic molecules are named, according to standards from IUPAC (the International Union of Pure and Applied Chemistry), in a prefix-stem-suffix fashion. This means that from the name alone it's possible to exactly reconstruct the molecular formula of the molecule. In order to work out the name of an organic compound given a formula, the following steps need to be taken:

\subsubsection{Work out the stem}
This is done by finding the longest carbon chain in the molecule (or this may be the longest chain with the functional group), and using the stem corresponding with its length (meth- eth-, prop-, but-, pent-, hex-, etc). Number this main chain so that the functionl group has the lowest number (or if this doesn't make sense, so that the alkyl groups have the lowest numbers).

\subsubsection{Work out the suffix}
The suffix will correspond to the type of compound, often the functional group but not always (for example in the case of alkanes). In some cases, if the functional group isn't on the end, it will also need a number to show which carbon it's on (for example in pentan-3-ol).

\subsubsection{Work out the prefix(es)}
Prefixes are needed for each alkyl group, or other groups of note such as halogens. They should have the number of the carbon they're on, hyphenated to the name of the group. In the case where there's more than one of the same group, the numbers of the carbons they're on can be listed by commas, and then a prefix can be added to the prefix for the group (eg 2,2-dichloro). Prefixes are conventionally added in alphabetical order of the group prefix (so ignoring the di- in the last example).

\subsection{Nomenclature reference}
Alkanes don't have a functional group. A molecule is an alkane if its a saturated hydrocarbon.

Alkenes have the functional group \chemfig{C=C}. The suffix -ene is used.

Halogenoalkanes have the funcional group \chemfig{-X} (where X is a halogen), and have a prefix of flouro / chloro / bromo / iodo as needed.

Alcohols have the functional group \chemfig{-O-H}. The suffix -ol is used, or if that's not possible due to another funcional group then the hydroxy- prefix can be used instead.

Aldehydes have the functional group \chemfig{-C([2]=O)-H}, and suffix -al.

Ketones have the functional group \chemfig{-C([2]-O)-}. The suffix -one is used, or if that's not possible then the prefix oxo- is used instead.

Carboxylic acids have functional group \chemfig{-C([2]=O)-O-H}, and use the suffix -oic acid.

Nitriles have the functional group \chemfig{-C~N}, and use the suffix -nitrile.

Amines have the functional group \chemfig{-N([2]-H)-H}. The use the suffix -amine, or if that's not possible then the prefix amino- is used instead.

Acyl chlorides have the functional group \chemfig{-C([2]=O)-Cl} and the suffix -oyl chloride.

Acid anydrides have the functional group \chemfig{-C([2]=O)-O-C([2]=O)-}, and the suffix -oic anydride.

Esters have the functional group \chemfig{-C([2]=O)-O-} and suffix -oate.

Amides have the functional group \chemfig{-C([2]=O)-N([2]-H)-H-} and the suffix -amide.

\subsection{Representations of molecules}
\subsubsection{Structural formulae}
Structural formulae use the smallest amount of detail possible to represent a molecule's structure. Alkyl branches are usually shown in brackets. For example, a molecule like $\ce{CH3CH2CH(CH3)2}$.

\subsubsection{Displayed formulae}
Displayed formulae depict the relative positions of atoms in a molecule, with bonds drawn in. An example would be:

\chemfig{H-C([2]-H)([-2]-H)-C([2]-H)([-2]-H)-C([2]-C)([-2]-C([-1]-H)([-2]-H)([-3]-H))-C([2]-H)([-2]-H)-H}

\subsubsection{Displayed formulae}
Displayed formulae 
\subsubsection{Skeletal diagrams}
Skeletal diagrams can be used to quickly and easily represent molecules. Lines are used to represent bonds, with there being assumed to be a carbon atom at the end of each line, and every carbon is assumed to be saturated with as many hydrogens as would give it four covalent bonds. This means that an intersection of two lines would usually represent a $\ce{CH2}$ group, while the end of a line would be a $\ce{CH3}$ group. Other functional groups are written in. Benzene rings are sometimes drawn as a hexagon with a circle inside.

\subsection{Structural isomerism}
Structural isomers are compounds with the same molecular formula, but different structural formulae. Examples would be longer alkenes which have the double bond across different carbons in the chain.

\subsection{Bond fission}
\subsubsection{Homolytic fission}
Homolytic fission of a covalent bond happens when the paired electrons return to their parent atoms (homo- because each atom gets the same amount of electrons back). The atoms both now have unpaired electrons and become what is called a free radical. Free radicals are extremely reactive and unstable.

\subsubsection{Heterolytic fission}
In heterolytic fission, both of the electrons from the covalent bond go to one of the atoms in the pair (hetero- because the atoms recieve different numbers of electrons). The atom that took both becomes a negative ion, while the atom that didn't get any becomes a positive ion.

\subsection{Types of reaction}
\subsubsection{Addition reactions}
In addition reactions, two reactants join to form one product. An example would be forming alcohols by reacting alkenes and water.

\subsubsection{Substitution reactions}
In substitution reactions, one or more atoms or groups are replaced by others. An example would be 1-bromopropane becoming propan-1-ol when an $\ce{OH-}$ group replaces a bromine.

\subsubsection{Elimination reactions}
Elimination reactions are when a small molecule is removed from a larger one. An example would be eliminating a water molecule from alcohols, in which an alkene and water are formed.

