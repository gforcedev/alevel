\section{Proof by induction}
Proof by induction is used for proofs with natural numbers. It works by a kind of ``domino'' effect. The steps are to show the given statement works for a base case (usually where $n=1$), and then show that if it works for one case, it must also work for the case directly above, thus proving the statement for all numbers above the base case.

\subsection{Example proof}
Proof that $\forall n \in \mathbb{Z}_1, \sum^{n}_{r=1}{r^2}=\frac{n}{6}(n+1)(2n+1)$ (referred to as *):

\subsubsection{Base case ($n=1$)}
\begin{ea}[rCl]
	\text{LHS} & = & \sum^{1}_{r=1}{r^2}
	\nonumber\\
			   & = & 1^2
	\nonumber\\
			   & = & 1
	\nonumber\\
	\text{RHS} & = & \frac{(1)}{6}((1)+1)((1)+2)
	\nonumber\\
			   & = & \frac{1}{6} \times 2 \times 3
	\nonumber\\
			   & = & 1
	\nonumber\\
			   & = & \text{LHS}
\end{ea}
So * holds for $n=1$.

\subsubsection{Assumption ($n=k$)}
Assume * works for some $k \in \mathbb{N}_1$, so:
\begin{ea}
	\sum^{k}_{r=1}{r^2}=\frac{k}{6}(k+1)(2k+1)
\end{ea}
(referred to as A).

\subsubsection{Inductive step ($n=k+1$)}
Aim to show * holds for $k+1$, so:
\begin{ea}[rCl]
	\sum^{(k+1)}_{r=1}{r^2} & = & \frac{(k+1)}{6}((k+1)+1)(2(k+1)+1)
	\nonumber\\
							& = & \frac{k+1}{6}(k+2)(2k+3)
\end{ea}

Then show it:
\begin{ea}[rCl]
	\sum^{(k+1)}_{r=1}{r^2} & = & \sum^{(k)}_{r=1}{r^2} + (k+1)^2
	\nonumber\\
	& = & \frac{k}{6}(k+1)(2k+1) + (k+1)^2
	\nonumber\\
	& = & (k+1)\left(\frac{k}{6}(2k+1) + (k+1)\right)
	\nonumber\\
	& = & (k+1)\left(2k^2+7k+6\right)
	\nonumber\\
	& = & (k+1)(k+2)(2k+3)
\end{ea}

\subsubsection{Conclusion}
* holds for $n=1$. If * holds for some case $n=k$, it also holds for $n=k+1$. Hence, by induction, * holds $\forall n \in \mathbb{N}_1$.

\subsection{Helpful question techniques}
\subsubsection{Series induction proofs}
For series induction proofs, the most important thing to keep in mind is to simplify and factorise as much as possible, as a general rule of thumb, at every stage. This is likely to make your working easier to follow and quicker to do.

\subsubsection{Matrix induction proofs}
Similarly, make sure to make sure you know exactly what you're aiming for by simplifying what you're aiming to show as much as possible.

\subsubsection{Divisibility results}
For divisibility results, the best thing to do is always to force your assumption into the inductive step ($k+1$ case), and then make up the difference. This should make it easier to prove the other part is divisible as required, because this is how the questions are designed.

