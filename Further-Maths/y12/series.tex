\section{Series}
\subsection{Basics}
Series can be split apart and constants can be taken out of the front, similar to integrals:
\begin{ea}
	\sum^{n}_{r=1}{2r^2-3r} = 2\sum^{n}_{r=1}{r^2} - 3\sum^{n}_{r=1}{r}
\end{ea}

If a series is from a number higher than one, it can be written as the series from 1 to $n$ minus the sum from 1 to $p$, where $p$ is 1 less than the starting number of the series ($p \in \mathbb{N}, p > 1$):
\begin{ea}
	\sum^{n}_{r=p}{r} = \sum^{n}_{r=1}{r} - \sum^{p-1}_{r=1}{r}
\end{ea}

\subsection{Important sums}
\subsubsection{Triangle numbers}
This formula needs to be memorised, it's not given in the formula booklet.
\begin{ea}
	\sum^{n}_{r=1}{r} = \frac{n}{2}(n+1)
\end{ea}

\subsubsection{Sum of squares}
This is given in the formula booklet.
\begin{ea}
	\sum^{n}_{r=1}{r^2} = \frac{n}{6}(n+1)(2n+1)
\end{ea}

\subsubsection{Sum of squares}
This is given in the formula booklet and is the square of the triangle numbers formula.
\begin{ea}
	\sum^{n}_{r=1}{r^3} = \frac{n^2}{4}(n+1)^2
\end{ea}
