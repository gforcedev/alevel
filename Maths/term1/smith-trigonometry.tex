\section{Trigonometry}
\begin{figure}[ht]
    \centering
    \incfig{a-triangle-labelled-for-the-sin-and-cosine-rules}
    \caption{A triangle labelled for the sin and cosine rules}
    \label{fig:a-triangle-labelled-for-the-sin-and-cosine-rules}
\end{figure}
\subsection{The cosine rule}
The cosine rule states that for any triangle with the angles labelled as $A$, $B$, and $C$, and the sides opposite them labelled as the corresponding lower-case letters, then:
\begin{equation}
	a^2=b^2+c^2-2bc\cos{A}
\end{equation}
This can also be written as:
\begin{equation}
	\cos{A}=\frac{b^2+c^2-a^2}{2bc}
\end{equation}

This can be used when you have a triangle with three sides labelled, or with two sides and the angle tha joins them labelled, in order to find the missing angle or side.

\subsection{The sine rule}
The sine rule states that for any triangle labelled in the same way as above, then:
\begin{equation}
	\frac{a}{\sin{A}}=\frac{b}{\sin{B}}=\frac{c}{\sin{C}}
\end{equation}
This can also be written as:
\begin{equation}
	\frac{\sin{A}}{a}=\frac{\sin{B}}{b}=\frac{\sin{C}}{c}
\end{equation}

It can be used when you have a triangle with one opposite pair labelled (eg $A$ and $a$), as well as one other value.

\subsection{The ambiguous case of the sin rule}
The ambiguous case of the sin rule occurs in triangles when you've been given an angle and two sides. This is because, due to the nature of a sine graph, there are two possible values that could work. A calculator will give one of the values, but you may need to check whether the question asked for an acute or obtuse angle, and possibly use the symmetry of a sine graph to find the value you're really looking for.
\begin{figure}[ht]
    \centering
    \incfig{the-ambiguous-case-of-the-sine-rule}
    \caption{The ambiguous case of the sine rule. Either of the dashed lines could be correct and so there are two possible angles.}
    \label{fig:the-ambiguous-case-of-the-sine-rule}
\end{figure}
