\section{Proof}
A proof is a demonstration that a mathematical statement is true, which only uses already known facts.

\subsection{Theorems vs conjectures}
This is a simple difference. A theorem is a statement that has been proved to be correct, while a conjecture is a statement that is yet to be proved.

\subsection{Types of proof}
\subsubsection{Proof by deduction}
This is the simplest kind of proof. It may often involve algebra, and will directly show the statement is true.
\subsubsection{Proof by exhaustion}
This works by proving every single possible case. As such, it is not always useable for all statements.
\subsubsection{Proof by contradiction}
This kind of proof works by assuming that the negation of a statement is true, and then doing something with the negation that leads to a false statement, thus proving the original non-negated version of the statement in question.
\subsubsection{Disproof by counterexample}
Fairly self-explanatory
\subsubsection{Proof by induction}
Don't need to know until later on in the course. Works by proving the statement for one case, and then proving that if it works for case $k$ it must also work for case $k+1$.

\subsection{Setting out proofs}
When setting out proofs, it is important to start with the true statement and work towards the statement in question. It's ok to have jottings that do this in the reverse order, but the proof must then be formally written out with the correct order of statements as well to gain the marks.
