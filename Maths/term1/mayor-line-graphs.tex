\section{Line Graphs}
\subsection{Working out a line between two points}
First use $m=\frac{\Delta y}{\Delta x}$, substituting in the coordinates of the two points to find $m$. Then, substitute this value of $m$, as well as the coordinates from one of the points, into $y=mx+c$. You can solve to find c, and then you have your line equation.
\subsection{Are they parallel?}
If they have the same gradient, then yes.
\subsection{Are they perpendicular?}
If their gradients multiply to -1, then yes.

\section{Lengths and Areas}
To find the distance between two points on a cartesian grid, we can create a right angled triangle and thereby use pythagoras to find the distance. That is to say, for the coordinates $A (x, y)$, and $B (x_1, y_1)$, then the distance between them is equal to $\sqrt{(x-x_1)^2+(y-y_1)^2}$.

\adjustbox{margin=1cm}{
    \pgfplotsset{}
    \begin{tikzpicture}[baseline]
        \begin{axis}[
            axis y line=center,
            axis x line=middle,
            axis equal,
            grid=both,
            xmax=5,xmin=-5,
            ymin=-5,ymax=5,
            xlabel=$x$,ylabel=$y$,
            width=10cm,
            anchor=center,
            ]
            \addplot coordinates {(1,4) (5,4)};
            \addplot coordinates {(1,4) (1,0)};
            \addplot coordinates {(1,0) (5,4)};
        \end{axis}
    \end{tikzpicture}
}

\section{Linear models}
Linear models can be used to represent the relationships between variables. For example, between a the mass on a spring and its extension, or between the amount of a product / duration of a service and its price.

\subsection{Is a linear model suitable?}
To find out whether a linear model is suitable for some given data, plot the data on the graph and see if the points fall roughly  on a straight line. If they do, an equation can be worked out by using the two given data points that are furthest away for the gradient.

\subsection{Explaining variables in linear models}
Questions will often require explaining what the gradient and the y-intercept refer to. The gradient can be described as the how much one variable changes \b{per} unit of change in the other. The y-intercept differs between questions. Examples might be the level of one variable when we started measuring, the flat fee charged for the service/product, or something else entirely. This just requires common sense.
