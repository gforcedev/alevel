\section{Graphs}
\subsection{Sketching cubic graphs}
When sketching a cubic graph, you need it in its factorised form. Then, you can find the y-intercept by multiplying all of the linear parts of each factorised term together, and find the x-intercepts in the same way in which you would for a quadratic. If a root is repeated, then the graph will just touch the $x$-axis at that point. Cubics will always follow the same general shape depending on whether they have a positive or negative coefficient of $x^3$. Examples could be:

\begin{figure}[ht]
    \centering
    \incfig{examples-of-cubic-graphs}
    \caption{Examples of cubic graphs. Left with positive $x^3$ coefficients, right with negative ones.}
    \label{fig:examples-of-cubic-graphs}
\end{figure}

\subsection{Quartic graphs}
Quartic graphs can have many different shapes (examples below). Sketching them is similar to sketching cubics.

\begin{figure}[ht]
    \centering
    \incfig{examples-of-quartic-graphs}
    \caption{Examples of quartic graphs. Bottom right with a negative $x^4$ coefficient.}
    \label{fig:examples-of-quartic-graphs}
\end{figure}

\subsection{Reciprocal graphs}
Reciprocal graphs are characterised by having \textbf{asymptotes}, which are lines which the graph gets infinitely close to without touching it. We can sketch these graphs by considering what will happen in each quadrant (often by trying the function with different inputs or outputs), and considering these asymptotes.

\begin{figure}[ht]
    \centering
    \incfig{examples-of-reciprocal-graphs}
    \caption{Examples of reciprocal graphs}
    \label{fig:examples-of-reciprocal-graphs}
\end{figure}

\subsubsection{Comparing reciprocal graphs}
Some questions might ask for multiple reciprocal graphs to be drawn on the same axes. For this scenario, it's best to think which graph will be higher at any given point. For example, $y=\frac{1}{x}$ will always be lower down than $y=\frac{2}{x}$ with the same $x$-value.

\subsection{Points of intersection of graphs}
The points at which $y=f(x)$ and $y=g(x)$ intersect are the solutions to the equation $f(x)=g(x)$. This means that we can solve the equation by sketching the graphs, as well as finding the points of intersection by solving the equation. Another possible question might be to say how many solutions an equation has without using the discriminant or some other method. In this case, sketching both sides of the equation is enough to deduce the answer.

\subsection{Transformations of graphs}
By manipulating graphs' equations in different ways, we can transform them in different ways on a grid.
\subsubsection{Translations}
$f(x)$ moves $a$ units up when it becomes $f(x)+a$. This is a translation by the vector $\begin{pmatrix}0 \\ a\end{pmatrix}$.

$f(x)$ moves $a$ units to the left when it becomes $f(x+a)$. This is a translation by the vector $\begin{pmatrix}-a \\ 0\end{pmatrix}$.

\subsubsection{Stretches}
When $f(x)$ becomes $af(x)$, it is stretched in the $y$ direction with a scale factor of $a$.

When $f(x)$ becomes $f(xa)$, it is stretched in the $x$ direction with a scale factor of $\frac{1}{a}$.

\subsubsection{Reflections}
Reflections are just special cases of stretches where the scale factor is $-1$. So, a reflection in the x-axis would be $f(x)$ becoming $-f(x)$, while a reflection in the y-axis would be $f(x)$ becoming $f(-x)$.
