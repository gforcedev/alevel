\section{Year 2 Differentiation}
\subsection{The chain rule}
The chain rule is used for differentiating composite functions.

In function notation:
\begin{IEEEeqnarray}{rCl}
	y & = & f(g(x))
	\nonumber\\
	\frac{dy}{dx} & = & f'(g(x)) \times g'(x)
\end{IEEEeqnarray}

In Leibnitz notation, where $y$ is a function of $u$ and $u$ is a function of $x$:
\begin{equation}
	\frac{dy}{dx} = \frac{dy}{du} \times \frac{du}{dx}
\end{equation}

It's easier to remember the catchphrase, or the steps for differentiating something using the chain rule, which is ``Differentiate the outer function keeping the inner one the same, and then multiply by the derivative of the inner function.''

\subsubsection{Proving the derivative of $e^{kx}$ is $ke^{kx}$}
Using the chain rule, we can differentiate $e^{kx}$ (not required, but a good example of the chain rule):
\begin{IEEEeqnarray}{rCl}
	y & = & e^{kx}
	\nonumber\\
	u & = & kx
	\nonumber\\
	\frac{du}{dx} & = & k
	\nonumber\\
	y & = & e^u
	\nonumber\\
	\frac{dy}{du} & = & e^u
	\nonumber\\
	\frac{dy}{du} & = & e^{kx}
	\nonumber\\
	\frac{dy}{dx} & = & \frac{dy}{du} \times \frac{du}{dx}
	\nonumber\\
	\frac{dy}{dx} & = & k \times e^{kx}
	\nonumber\\
	\frac{dy}{dx} & = & ke^{kx}
\end{IEEEeqnarray}

\subsection{Differentiating exponentials and natural logarithms}
\subsubsection{Differentiating natural logarithms}
The derivative of $\ln{x}$ is $\frac{1}{x}$. The proof for this (not required), uses the trick of rearranging to make $x$ the subject, differentiating for $y$, and then flipping $\frac{dx}{dy}$ upside down (bringing the value to the power of -1) in order to get $\frac{dy}{dx}$.

\subsubsection{Differentiating exponentials}
The derivative of $a^x$ is $a^x\ln{a}$. The proof for this (again, not requred), works by bringing $e$ to the power of the natural logarithm of the original fuction, and then using log rules to rearrange to something easier to differentiate:
\begin{equation}
	y = a^x = e^{\ln{a^x}} = e^{(\ln{a})x}
	\nonumber
\end{equation}
We can now use that the derivative of $e^{kx}$ is $ke^{kx}$:
\begin{equation}
	\frac{dy}{dx} = (\ln{a})e^{(\ln{a})x} = (\ln{a})e^{\ln{a^x}} = a^x\ln{a}
\end{equation}

We can use this in conjunction with the chain rule to prove that the derivative of $y=a^{kx}$ is $ka^{kx}\ln{a}$

\subsection{The product rule}
The product rule is used for differentiating two functions multiplied together. In function notation: if $y=f(x)\times g(x)$, then
\begin{equation}
	\frac{dy}{dx}=f(x)\times g'(x) + g(x)\times f'(x)
\end{equation}

In Leibnitz notation: if $y=uv$, then
\begin{equation}
	\frac{dy}{dx}=u\frac{dv}{dx}+v\frac{du}{dx}
\end{equation}

In terms of a catchphrase, then to apply the product rule take the first function multiplied by the derivative of the second, and add the second function multiplied by the derivative of the first.

The proof for this involves setting up a differentiation from first principles, and then both subtracting and adding $f(x+h)g(x)$:
\\\\
Let $y=f(x)g(x)$
\begin{IEEEeqnarray}{rCl}
	\frac{dy}{dx} & = & \lim_{h\to 0} \frac{f(x+h)g(x+h)-f(x)g(x)}{h}
	\nonumber\\
	& = & \lim_{h\to 0}\frac{f(x+h)g(x+h)-f(x+h)g(x)+f(x+h)g(x)-f(x)g(x)}{h}
	\nonumber\\
	& = & \lim_{h\to 0}f(x+h)\left(\frac{g(x+h)-g(x)}{h}\right)+g(x)\left(\frac{f(x+h)-f(x)}{h}\right)
	\nonumber\\
	& = & \lim_{h\to 0}\left[f(x+h)\left(\frac{g(x+h)-g(x)}{h}\right)+\right]+\lim_{h\to 0}\left[g(x)\left(\frac{f(x+h)-f(x)}{h}\right)\right]
	\nonumber\\
	& = & \lim_{h\to 0}f(x+h)\times\lim_{h\to 0}\left[\frac{g(x+h)-g(x)}{h}\right] + \lim_{h\to 0}g(x)\times\lim_{h\to 0}\left[\frac{f(x+h)-f(x)}{h}\right]
	\nonumber\\
	& = & f(x)g'(x)+g(x)f'(x)
\end{IEEEeqnarray}

\subsection{The quotient rule}
The quotient rule is used for differentiating two functions, one divided by the other. In Leibnitz notation, if $y=\frac{u}{v}$:
\begin{equation}
	\frac{dy}{dx} = \frac{v\frac{du}{dx}-u\frac{dv}{dx}}{v^2}
\end{equation}

In words, then it's the bottom times the derivative of the top, minus the top times the derivative of the bottom, all over the bottom squared.

The proof for this involves using the product rule for for $uv^{-1}$,
because, by the chain rule, we can use that $\frac{dv}{dx}v^{-1}=-v^{-2}\times\frac{dv}{dx}$:
\begin{IEEEeqnarray}{rCl}
	\frac{dy}{dx} & = & u\times \frac{d}{dx}(v^{-1})+v^{-1}\times\frac{du}{dx}
	\nonumber\\
	& = & u\times \left(-v^{-2}\times \frac{dv}{dx}\right)+v^{-1}\times\frac{du}{dx}
	\nonumber\\
	& = & \frac{-u}{v^2}\times \frac{dv}{dx}+v^{-1}\times\frac{du}{dx}
	\nonumber\\
	& = & \frac{v\frac{du}{dx}-u\frac{dv}{dx}}{v^2}
\end{IEEEeqnarray}
