\section{Radians}
Radians are a unit of angle measurement. They are based on a sector with arc length the same as the circle's radius. The angle of that sector is one radian. There are $2\pi$ radians in 360 \textdegree.

Some other useful angle conversions:
\begin{table}[ht]
\begin{tabular}{|l|l|}
\hline
Degrees & Radians          \\ \hline
360     & $2\pi$           \\ \hline
270     & $\frac{3\pi}{2}$ \\ \hline
180     & $\pi$            \\ \hline
90      & $\frac{\pi}{2}$  \\ \hline
60      & $\frac{\pi}{3}$  \\ \hline
45      & $\frac{\pi}{4}$  \\ \hline
30      & $\frac{\pi}{6}$  \\ \hline
\end{tabular}
\end{table}

These would be pronounced, for example, ``$\pi$ by 2'' or ``$\pi$ by 3''.

\subsection{Arc length}
To work out arc length in degrees, you divide the angle by 360, and multiply that fraction by the circumference. For A-level, the same thing applies, but first you have to convert the angle from radians to degrees (multiply by $\frac{180}{\pi}$). Cancelling down this formula gives that the formula for arc length in radians is much nicer:
\begin{equation}
	\frac{\theta \times \frac{180}{\pi}}{360}\times 2\pi r = r\theta
\end{equation}

\subsection{Sector area}
Similar to arc length, we can use the same formula as degrees, but convert to radians first and then cancel to give a nicer formula for radians:
\begin{equation}
	\frac{\theta \times \frac{180}{\pi}}{360}\times \pi r^2 = \frac{1}{2}r^2\theta
\end{equation}

\subsection{Small angle approximations of trig functions}
When using radians, the trigonometric functions tend to simpler functions as $\theta$ gets small.

If $\theta$ is small, then:
\begin{IEEEeqnarray}{rCl}
	\sin{\theta} & \approx & \theta
	\nonumber\\
	\cos{\theta} & \approx & 1-\frac{\theta^2}{2}
	\nonumber\\
	\tan{\theta} & \approx & \theta
\end{IEEEeqnarray}

This can be shown by comparing the graphs and focusing on the area where $x$ gets close to 0. This is useful because if $\theta$ is small, we can substitute in its the small angle approximation of any trig functions and often cancel to a much nicer expression in questions.
