\section{Functions and graphs}
\subsection{The modulus function}
The modulus function, written as $|x|$, makes the number it is applied to non-negative.
For graphs, this means that every part of the graph that would be below the $x$-axis gets reflected to above it.
\begin{figure}[ht]
    \centering
    \incfig{linear-modulus-graph}
    \caption{linear modulus graph}
    \label{fig:linear-modulus-graph}
\end{figure}

So: 

If $f(x) \ge 0$, then $|f(x)|=f(x)$.

If $f(x)<0$, then $|f(x)|=-f(x)$.

\subsubsection{Modulus of the input}
$y=f(|x|)$ means that all of the negative inputs to the function are turned positive. This means that to sketch the function $y=f(|x|)$ from $y=f(x)$, sketch the part to the right of the y-axis, and then mirror it in the left-hand quadrants.

\subsection{Solving equations with modulus}
To solve equations with modulus functions, the first thing to do is to sketch the graphs of both sides. This will show how many intersections are expected. Then, two separate equations need to be set up: one with the thing inside the modulus as written, and one with it multiplied by $-1$. So $y=3|x-1|$ would be solved as both $y=3(x-1)$ and $y=3(1-x)$.

If asked to find a value in a modulus equation which would make it have a certain number of solutions, sketch with the values that are given, and then think about what the gradient or $y$-intercept would have to be to give the number of intersections required.

\subsection{Functions and mappings}
A mapping transforms one set of numbers onto another. Functions are defined as mappings which have a distinct output for every input. They can be one-to-one (every input corresponds to one output, and is the only input that can lead to that output), or many-to-one (there is more than one input that can give the same output, for example quadratic functions). However, functions cannot be one-to-many (inputs having more than one possible output): this would be a mapping. In addition, functions must have a defined output for every possible input (unlike, for example, $y=\sqrt x$). This means that while all functions are mappings, not all mappings are functions.

\subsubsection{Domains and ranges}
The domain of a function is all of the possible inputs it can have. The range of a function is all of the possible outputs it can have. Sometimes, a mapping can be made into a function by restricting its domain.

\subsubsection{Piece-wise functions}
Some functions can be defined piece-wise, with different operations happening depending on the input. When drawing piece-wise functions on graphs, a filled-in circle should be used at the end of a line if the output can actually take that value based on the domain, and an outlined one should be used if it can't.

\subsection{Composite functions}
Composite functions are when two or more functions are applied in succession. For example, $fg(x)$, which represents doing $f(g(x))$.

\subsubsection{Domains and ranges of composite functions}
Composite functions have domains restricted by both of their components, so they should be combined in the most restrictive way possible.

\subsection{Inverse functions}
The inverse function $f^{-1}(x)$ does the opposite to $f(x)$. The range of a function is the domain of its inverse, and the graphs are reflections in $y=x$. Inverse functions can only exist for one-to-one functions, because the inverse of a one-to-many function would be a many-to-one mapping.

\subsection{Combinations of transformations of graphs}
When performing multiple transformations on graphs, the best way is to track key points (turning points and intersections with axes) across the transformations, and then considering whether the line has been reflected at all.

\subsubsection{List of graph transformations}
When $y=f(x)$ becomes:
\begin{itemize}
	\item $y=f(x+a)$, it gets translated $a$ units to the left.
	\item $y=f(x)+a$, it gets translated $a$ units up.
	\item $y=af(x)$, it gets stretched in the $y$-direction scale factor $a$.
	\item $y=f(ax)$, it gets stretched in the $x$-direction scale factor $a^{-1}$.
\end{itemize}

