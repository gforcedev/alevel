\section{Hypothesis testing}
Conducting a hypothesis test follows a series of steps. First, the test statistic is identified. This is the result of the experiment that's being carried out. Next, two hypotheses are formed. The null hypothesis, $H_0$, is assumed to be correct. The alternative hypothesis, $H_1$, is what you're checking to see there's statistically significant evidence to support, so that you can reject the null hypothesis, based on the experimental data collected.

\subsection{An example hypothesis test}
To test whether a coin is biased towards heads, the test statistic would be the number of heads thrown. The null hypothesis would be that there was half a chance of heads, and the alternative hypothesis would be that there was more than half a chance of heads:
\begin{IEEEeqnarray}{rr}
	H_0: & p=0.5
	\nonumber\\
	H_1: & p>0.5
	\nonumber
\end{IEEEeqnarray}

What we need to do now is run a test, and see how likely the result was to happen. Consider if the coin was flipped 10 times, and landed on heads for 7 of them. We assume the null hypothesis to be true, and so assume that $x \sim Bin(10, 0.5)$, and then use a calculator to work out the probability of 7 (or more) out of the 10 throws being heads if this was the case. In order to do this, we will need to do 1 minus the probability of 6 or lower, which gives an answer of $0.171875$.

We then compare this to a predetermined significance level, which is given in the question, with $5\%$ being convention. In this case, $0.171875>0.5$, so there is not enough evidence to reject the null hypothesis.

\subsection{Some definitions}
The critical region, or rejection region, is the set of values which would be unlikely enough to provide evidence to reject the null hypothesis.

The critical value is the value on the inside edge of the critical region.

The acceptance region is the set of values which wouldn't cause the null hypothesis to be rejected.
\subsection{One- and two-tailed tests}
A one-tailed test would have $H_1$ be either $p \leq a$, or $p \geq a$, where $a$ is some constant, and so would have a critical region with all values next to each other.

A two-tailed test would have $H_1$ be $p \neq a$, and so the critical region would contain values above or below, and the significance value would be split between the two sets of values.
\begin{figure}[ht]
    \centering
    \incfig{one-tailed-and-two-tailed-test}
    \caption{one-tailed and two-tailed test}
    \label{fig:one-tailed-and-two-tailed-test}
\end{figure}
