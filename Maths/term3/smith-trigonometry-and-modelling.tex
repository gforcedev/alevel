\section{Trigonometry and modelling}
\subsection{Addition angle formulae}
The double angle formulae can be used to split trigonometric functions of two terms added or subtracted into more trigonometric functions, but all containing only one term. They are:
\begin{IEEEeqnarray}{rCl}
	\sin(A\pm B) & \equiv & \sin{A} \cos{B} \pm \cos{A} \sin{B}
	\\
	\cos(A\pm B) & \equiv & \cos{A} \cos{B} \mp \sin{A} \sin{B}
	\\
	\tan(A\pm B) & \equiv & \frac{\tan{A} \pm \tan{B}}{1 \mp \tan{A} \tan{B}}
\end{IEEEeqnarray}

A common use case for the double angle formlae is splitting one term into the sum of two in order to get an exact value out of a trig function. For example, finding the exact value of $\sin{15^\circ}$ by splitting it into $\sin{45^\circ}-\sin{30^\circ}$.

\subsection{Double angle formulae}
The double angle formulae are special cases of the addition angle formulae (ie. the case when $A$ and $B$ are the same). They simplify the formulae to:

\begin{IEEEeqnarray}{rCl}
	\sin(2A) & \equiv & 2\sin{A} \cos{A}	
	\\
	\cos(2A) & \equiv & \cos^2{A} - \sin^2{A}
	\nonumber\\
			 & \equiv & 1-2\sin^2{A}
	\nonumber\\
			 & \equiv & 2\cos^2{A}-1
	\\
	\tan(2A) & \equiv & \frac{2\tan{A}}{1 - \tan^2{A}}
\end{IEEEeqnarray}

\subsection{$R\cos{(\theta\pm\alpha)}$ and $R\sin{(\theta\pm\alpha)}$}
This technique can be used to combine two separate trogonometric functions into one. It is best demonstrated on the example question of writing $5\sin{\theta}+12\cos{\theta}$ in the form $Rsin{(\theta+\alpha)}$. First of all, we expand the target with the addition angle formulae.

\begin{IEEEeqnarray}{rCl}
	5\sin{\theta}+12\cos{\theta} & \equiv & R\sin{(\theta+\alpha)}
	\nonumber\\
	5\sin{\theta}+12\cos{\theta} & \equiv & R\sin{\theta}\cos{\alpha} + R\cos{\theta}\sin{\alpha}
	\nonumber
\end{IEEEeqnarray}

Next, we can form two equations by equating coefficients of $\sin{\theta}$ and $\cos{\theta}$:
\begin{IEEEeqnarray}{rCl}
	5 & = & R\cos{\alpha}
	\nonumber\\
	12 & = & R\sin{\alpha}
	\nonumber
\end{IEEEeqnarray}

To find $R$, we can square both sides of both equations (we know this won't affect answers as we assume $R>0$), and then add them:
\begin{IEEEeqnarray}{rCl}
	25 & = & R^2\cos^2{\alpha}
	\nonumber\\
	144 & = & R^2\sin^2{\alpha}
	\nonumber\\
	169 & = & R^2\cos^2{\alpha} + R^2\sin^2{\alpha}
	\nonumber\\
	169 & = & R^2(\cos^2{\alpha} + \sin^2{\alpha})
	\nonumber\\
	169 & = & R^2
	\nonumber\\
	13 & = & R
	\nonumber
\end{IEEEeqnarray}

In practice, we can simply use the coefficients of $\sin{\theta}$ and $\cos{\theta}$ from the question the two short sides of a pythagorean triple to find $R$ without needing to go through the entire process.

To find $\alpha$, we divide the two equations, and then inverse the resulting $\tan$ to get our answer.
\begin{IEEEeqnarray}{rCl}
	\frac{12}{5} & = & \frac{R\sin{\alpha}}{R\cos{\alpha}}
	\nonumber\\
	\theta & = & \arctan{\frac{12}{5}}
	\nonumber\\
	\theta & \approx & 67.4
	\nonumber
\end{IEEEeqnarray}

So, as a final answer:
\begin{equation}
	5\sin{\theta}+12\cos{\theta} \equiv 13\sin{(\theta+67.4)}
\end{equation}

In questions, this technique is often used at the start in order to end up solving an equation that you need to put in this form.
