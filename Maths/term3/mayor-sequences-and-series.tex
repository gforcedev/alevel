\section{Sequences and series}
\subsection{Arithmetic sequences}
Arithmetic, or linear sequences, are sequences where the difference between consecutive terms is constant.

We describe arithmetic sequences with first term $a$, and common difference $d$. This makes the general arithmetic sequence $a, (a+d), (a+2d)$, etc. The $n$th term of a general arithmetic sequence is $a+(n-1)d$.

\subsection{Arithmetic series}
A series is the sum of numbers in a sequence, and so an arithmetic series is the sum of terms in an arithmetic sequence.

\subsubsection{Sum to $n$ of an arithmetic series}
Consider this method of summing the numbers from 1 to 100: First, write them all out forward:
\begin{equation}
	S_{100}=1+2+3+...+100
	\nonumber
\end{equation}
Then, write them out backward:
\begin{equation}
	S_{100}=100+99+98+...+1
	\nonumber
\end{equation}

Adding terms above each other will always give 101, so:
\begin{IEEEeqnarray}{rCl}
	2S_{100} & = & 100\times101
	\nonumber\\
	S_{100} & = & \frac{100\times101}{2}
	\nonumber\\
	S_{100} & = & 5050
\end{IEEEeqnarray}

This also works for the general case, to give the general sum to $n$ of an arithmetic series:
\begin{IEEEeqnarray}{rCl}
	S_n & = & a + (a+d) + ... + (a+(n-1)d)
	\nonumber\\
	S_n & = & (a+(n-1)d) + (a+(n-2)d) + ... + a
	\nonumber
\end{IEEEeqnarray}
So the first and last term, second and penultimate term, and so on all sum to $2a+(n-1)d$:
\begin{IEEEeqnarray}{rCl}
	2S_n & = & n(2a+(n-1)d)
	\nonumber\\
	S_n & = & \frac{n}{2}(2a+(n-1)d)
\end{IEEEeqnarray}
This proof of the general sum to $n$ of an arithmetic series is examinable.

As the $n$th term is equal to $a+(n-1)d$, then that can be substituted into this formula to give that $S_n=\frac{n}{2}(a+l)$ where $l$ is the last term in the series.

\subsection{Geometric sequences}
Geometric sequences are sequences in which there is a common ratio to get from one term to the next. They can be represented with first term $a$, and common ratio $r$.

The general geometric sequence would be $a, (ar), (ar^2), (ar^3)$, etc. This makes the general $n$th term for a geometric sequence $ar^{n-1}$.

An extremely useful process for solving questions involving arithmetic sequences is dividing two adjacent terms to find $r$ (or dividing any two terms to get some power of it).

\subsection{Geometric series}
Geometric series are the sums of geometric sequences.

\subsubsection{Sum of geometric series}
The proof for the sum to $n$ of a general geometric series is similar to the one for an arithmetic series (and similarly examinable at A-level). However, instead of writing the sequence out backwards and adding the two together, you multiply by $r$ and then subtract. Everything apart from the first and last terms then cancel out and the solution is easy to get to after that point.
\begin{IEEEeqnarray}{rCl}
	S_n & = & a + (ar) + (ar^2) + ... + (ar^{n-1})
	\nonumber\\
	rS_n & = & ar + (ar^2) + (ar^3) + ... + (ar^n)
	\nonumber\\
	rS_n-S_n & = & a(1-r^n)
	\nonumber\\
	S_n(1-r) & = & a-(1-r^n)
	\nonumber\\
	S_n & = & \frac{a(1-r^n)}{1-r}
\end{IEEEeqnarray}

\subsubsection{Convergent and divergent sequences}
Convergent geometric series are geometric series where the terms get closer and closer to 0. This happens when $|r|<1$.

Divergent geometric series are geometric series where the terms get further away from 0. This happens when $|r|>1$.

\subsubsection{Sum to infinity of convergent geometric series}
If a geometric series is convergent, then as $n\to\infty$, $S_n\to k$, where $k$ is a constant. We can find this constant by working out the limit as $n\to\infty$ of the formula for the sum of the geometric series:

\begin{IEEEeqnarray}{rCl}
	S_\infty & = & \lim_{n\to\infty}\frac{a(1-r^n)}{1-r}
	\nonumber\\
	S_\infty & = & \frac{a}{1-r}
\end{IEEEeqnarray}

This is because as $n\to\infty$, $r^n\to 0$.

For divergent series, this isn't applicable because as $n\to\infty$, $S_n\to\infty$.

\subsection{Sigma notation}
Sums of series can be written in sigma notation. A capital sigma is used, with the variable and the number it starts at underneath, and the number it finishes at (inclusive) above. For example:
\begin{equation}
	\sum_{r=1}^{20}4r+1
	\nonumber
\end{equation}

\subsection{Recurrence relations}
A recurrence relation is a way of expressing terms of a sequence in terms of the previous term. The are often in the form $U_{n+1}=f(U_n)$. $U_1$ (or, at least one value somewhere in the sequence) also needs to be specified.

\subsubsection{Increasing and decreasing sequences}
Sequences are increasing if $U_{n+1}>U_n \forall n \in N$.

Sequences are decreasing if $U_{n+1}<U_n \forall n \in N$.

Note that there are some sequences that are neither, for example geometric sequences with a negative common ratio.

\subsubsection{Periodic sequences}
Sequences are periodic if their terms repeat in a cycle. The number of terms in one complete cycle is called the period, or the order.

\subsection{Modelling with sequences and series}
Moddeling questions involving sequences and series often use the trick of having the first year actually be the 0th term in the sequence, or maybe the second. A useful tip for sequences modelling questions is to be doubly sure exactly which term in the sequence you need to work out.
