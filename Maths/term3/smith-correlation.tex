\section{Correlation}
\subsection{Bivariate data}
Bivariate data is data which maps one variable to another. An example might be visibility vs weather humidity. Correlation describes the nature of the linear relationship between these two variables.

Correlation can be strong or weak depending on how exactly the pattern is followed, and can be positive (one variable increases with the other), or negative (one variable decreases with the other). Data can also have no correlation.

\subsection{Linear regression}
Linear regression is used to accurately draw a line of best fit through the data. One way of doing this is using the least squares regression line, which minimises the value of the sum of the squares of the vertical distances of points away from the line.

The form of the linear regression line for $y$ on $x$ is $y=a+bx$, and can be found with a calculator. The coefficient $b$ tells whether the data is positively or negatively correlated, and how steep the line will be.

\subsection{Common issues with using regressions to make observations}
\subsubsection{Correlation $\ne$ causation}
If there is no proven causal link, then a causation is not enough to say that one metric affects the other. This can also be thought of as only being able to use the regression line to make predictions for values of the dependent variable rather than of the independent variable.

\subsubsection{Extrapolation vs interpolation}
Extrapolation is using trends in your data to predict things outside of the range, and is much less reliable than interpolation, which is the same but predicts things which are inside the range of the data.
