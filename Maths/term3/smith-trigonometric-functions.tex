\section{Trigonometric functions}
\subsection{Reciprocal trig functions}
There are three reciprocal trig functions, secant ($\sec$), cosecant ($\mathrm{cosec}$), and cotan ($\cot$). They can be remembered because their third letter matches the first letter of their corresponding normal trig function:
\begin{IEEEeqnarray}{rCl}
	\mathrm{cosec}{x} & = & \frac{1}{\sin{x}}
	\nonumber\\
	\sec{x} & = & \frac{1}{\cos{x}}
	\nonumber\\
	\cot{x} & = & \frac{1}{\tan{x}}
\end{IEEEeqnarray}

\subsubsection{Graphs of reciprocal trig functions}
The ranges of the reciprocal trig functions of $\sin$ and $\cos$ are that the output has to be greater than $1$ or less than $-1$, while $\cot{x}$ looks similar to a tan graph but flipped in the $x$-axis and going through the $y-axis$ first at $\pm\frac{\pi}{2}$.
\begin{figure}[ht]
    \centering
    \incfig{reciprocal-trig-graphs}
    \caption{reciprocal trig graphs}
    \label{fig:reciprocal-trig-graphs}
\end{figure}

\subsubsection{Trig identities involving reciprocal trig functions}
From the trigonometric identity that $\sin^2\theta + \cos^2\theta = 1$, we can divide by either $\sin^2\theta$ or $\cos^2\theta$ to give two new identities:
\begin{IEEEeqnarray}{rCl}
	1+\frac{\cos^2\theta}{\sin^2\theta} & \equiv & \frac{1}{sin^2\theta}
	\nonumber\\
	1+\cot{\theta} & \equiv & \mathrm{cosec}^2\theta
	\\
	\frac{\sin^2\theta}{\cos^2\theta} + 1 & \equiv & \frac{1}{\cos^2\theta}
	\nonumber\\
	\tan^2\theta + 1 & \equiv & \sec^2\theta
\end{IEEEeqnarray}

