\section{Trigonometric functions}
\subsection{Reciprocal trig functions}
There are three reciprocal trig functions, secant ($\sec$), cosecant ($\mathrm{cosec}$), and cotan ($\cot$). They can be remembered because their third letter matches the first letter of their corresponding normal trig function:
\begin{IEEEeqnarray}{rCl}
	\mathrm{cosec}{x} & = & \frac{1}{\sin{x}}
	\nonumber\\
	\sec{x} & = & \frac{1}{\cos{x}}
	\nonumber\\
	\cot{x} & = & \frac{1}{\tan{x}}
\end{IEEEeqnarray}

\subsubsection{Graphs of reciprocal trig functions}
The ranges of the reciprocal trig functions of $\sin$ and $\cos$ are that the output has to be greater than $1$ or less than $-1$, while $\cot{x}$ looks similar to a tan graph but flipped in the $x$-axis and going through the $y-axis$ first at $\pm\frac{\pi}{2}$.
\begin{figure}[ht]
    \centering
    \incfig{reciprocal-trig-graphs}
    \caption{reciprocal trig graphs}
    \label{fig:reciprocal-trig-graphs}
\end{figure}
