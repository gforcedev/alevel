\section{A-level prep (holiday work)}
Work from \href{https://beechencliffmaths.weebly.com/as-prep-edexcel.html}{the weebly page}.
\subsection{Laws of indices}
When multiplying indices, add the exponents. When dividing indices, subtract them.
\begin{IEEEeqnarray}{rCl}
    x^m\times x^n & = & x^{m+n}
    \\
    \frac{x^m}{x^n} & = & x^{m-n}
\end{IEEEeqnarray}

For nested indices, multiply the exponents.
\begin{equation}
    (x^m)^n = x^{mn}
\end{equation}

Fractional indices take the form $\frac{\textrm{power}}{\textrm{root}}$.
\begin{equation}
    x^{\frac{m}{n}} = \sqrt[n]{x^m}
\end{equation}

Negative indices indicate ``one over''.
\begin{equation}
    x^{-m} = \frac{1}{x^m}
\end{equation}

Anything raised to the power 0 is 1.
\begin{equation}
    x^0 = 1
\end{equation}

\subsection{Surds}

\subsubsection{Key surd rules}

\paragraph{Multiplying}
\begin{equation}
    \sqrt{x} \sqrt{y} = \sqrt{xy}
\end{equation}

\paragraph{Dividing}
\begin{equation}
    \frac{\sqrt{x}}{\sqrt{y}} = \sqrt{\frac{x}{y}}
\end{equation}

\subsubsection{Simplifying surds}
To simplify a surd, find the largest square number that divides it.
\begin{IEEEeqnarray}{rCl}
    \sqrt{50} & = & \sqrt{25\times 2}\nonumber
    \\
    & = & 25\sqrt{2}
\end{IEEEeqnarray}

\subsubsection{Rationalising the denominator}
To rationalise the denominator of a fraction containing a surd, multiply by a fraction with that surd top and bottom (which is equal to 1).
\begin{IEEEeqnarray}{rCl}
    \frac{1}{\sqrt{3}} & = & \frac{1}{\sqrt{3}}\times \frac{\sqrt{3}}{\sqrt{3}} \nonumber
    \\
    & = & \frac{\surd{3}}{3}
\end{IEEEeqnarray}

For more complicated instances, the difference of two squares can be used.
\begin{IEEEeqnarray}{rCl}
    \frac{3}{2+\sqrt{5}} & = &
    \frac{3}{2+\sqrt{5}}\times \frac{2-\sqrt{5}}{2-\sqrt{5}}
    \nonumber \\
    & = & \frac{3(2-\sqrt{5})}{(2+\sqrt{5})(2-\sqrt{5})}
    \nonumber \\
    & = & \frac{6 - 3\sqrt{5}}{-1}
    \nonumber \\
    & = & 3\sqrt{5}-6
\end{IEEEeqnarray}

\subsection{Completing the Square}
Completing the square rearranges a quadratic from the form $ax^2+bx+c$ into the form $p(x+q)^2+r$. If $a \neq 1$, then this is done by factorising $a$ out of the first 2 terms.

\begin{IEEEeqnarray}{rCl?s}
    2x^2-5x+1 & = & 2\left(x^2+\frac{5}{2}x\right)+1 & // factorise out $a$
    \nonumber \\
    & = & 2\left[\left(x-\frac{5}{4}\right)^2-\left(\frac{5}{4}\right)^2\right]+1 & // CTS inner part
    \nonumber \\
    & = & 2\left(x-\frac{5}{4}\right)^2-\frac{25}{8}+1 & // multiply out
    \nonumber \\
    & = & 2\left(x-\frac{5}{4}\right)^2-\frac{7}{8}+1 & // simplify
\end{IEEEeqnarray}