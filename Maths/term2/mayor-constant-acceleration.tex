\section{Constant acceleration}
\subsection{Displacement-time graphs}
On s-t graphs, the gradient represents the velocity. The average velocity is the total displacement divided by the total time.
\begin{figure}[ht]
    \centering
    \incfig{examples-of-displacement-time-graphs}
    \caption{Examples of displacement-time graphs}
    \label{fig:examples-of-displacement-time-graphs}
\end{figure}

\subsection{Velocity-time graphs}
On v-t graphs, the gradient represents acceleration, and the area underneath represents distance travelled.
\begin{figure}[ht]
    \centering
    \incfig{examples-of-velocity-time-graphs}
    \caption{Examples of velocity-time graphs}
    \label{fig:examples-of-velocity-time-graphs}
\end{figure}

\subsection{Constant Acceleration equations}
The constant acceleration equations (also known as the suvat equations because of the variable names used), apply to any body with constant acceleration. $s$ is the displacement, $u$ is the initial velocity, $v$ is the final velocity, $a$ is the acceleration, and $t$ is the time.

Here's a list of the suvat equations:
\begin{IEEEeqnarray}{rCl}
	v & = & u + at
	\\
	s & = & \frac{1}{2}t(v+u)
	\\
	v^2 & = & u^2 + 2as
	\\
	s & = & ut + \frac{1}{2}at^2
	\\
	s & = & vt - \frac{1}{2}at^2
\end{IEEEeqnarray}

\subsection{Deriving the suvat equations}
The suvat equations can be derived form displacement-time and velocity-time graphs.
\subsubsection{Deriving $v=u+at$}
$v=u+at$ comes from the fact that the gradient on a displacement-time graph represents the acceleration.
\subsubsection{Deriving $s=\frac{1}{2}t(v+u)$}
This equation comes from the fact that the area under a velocity-time graph represents the displacement.
\subsubsection{Deriving the other suvat equations}
The other three suvat equations can be found by solving these first two simultaneously, using substitution to eliminate either $t$, $v$, or $u$.

\subsection{Using the suvat equations}
When solving problems using suvat equations, a good first step is to make a note of all of the suvat variables that are known (remember, the key part of many questions is that $a$ can be acceleration due to gravity, so 9.8 can be assumed). After that then looking through which of the five equations concern the variables known/needed should give a good idea of which one(s) should be used to solve the problem.
