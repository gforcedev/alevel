\section{Integration}
Integration is the opposite of differentiation, meaning that we follow the opposite steps to perform it. When differentiating, we multiply the term by the power of the variable in question, and then reduce this power by 1. Correspondingly, when integrating, we add 1 to the power of the variable, and then divide the whole term by this new power.

\subsection{The constant of integration}
When integrating any function, we must include the constant of integration, $c$. This is because we cannot know what constant was in the original function (as any constant differentiates to 0).

\subsection{A formal definition of integration}
If $f'(x) = ax^n$, then $f(x)=\frac{ax^{n+1}}{n+1}+c$ ($n\ne 1$).

\subsection{Finding the constant of integration}
If given a gradient function $f'(x)$, and a point that the line $y=f(x)$ passes through, it is possible to find the constant of integration (and thus the original function), by integrating, and substituting in the coordinates to give an equation where the only unknown is $c$.

\subsection{The integration symbol}
The symbol for integration looks like an elongated `S'. For example, $\int x^{\frac{1}{2}}+2x^3 dx$ means to integrate $x^{\frac{1}{2}}+2x^3$, with the $dx$ meaning to integrate for $x$, rather than for any other constant that might be in the function.

\subsection{Definite and indefinite integrals}
Indefinite integrals are what you get when you simply integrate a function. They give another function as an answer, and this function always contains the constant of integration $c$.

Definite integrals are different - they give a numerical answer, and are found by integrating a function, evaluating it at two limits, and taking the result at the lower limit away from the result at the upper limit. Because we're evaluating with two different arguments and subtracting one from the other, the instance $c$ from each answer will cancel, and so we can ignore it in this case. The limits are written at either end of the $\int$ symbol, like in $\int\limits_{0}^{2}$.

\subsection{Setting out definite integrals}
When working out definite integrals, they should be set out like this:
\begin{IEEEeqnarray}{Cl}
	& \int\limits_{2}^{4}x^2+5 dx
	\nonumber\\
	= & \left[\frac{x^3}{3}+5x\right]\limits_{2}^{4}
	\nonumber\\
	= & \left(\frac{(4)^3}{3}+5(4)\right)-\left(\frac{(2)^3}{3}+5(2)\right)
	\nonumber\\
	= & \frac{86}{3}
\end{IEEEeqnarray}

\subsection{The fundamental theorem of calculus}
Let a function $f(x)$ have derivative $f'{x}$ for all $x \in [a,b]$
