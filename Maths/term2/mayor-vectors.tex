\section{Vectors}
\subsection{Vector basics}
Vectors have both direction, and magnitude.

If two vectors are equal, this means that both the direction and the magnitude are the same.

Vectors $\vec{XY}=-\vec{YX}$ because they have equal magnitude but opposite direction.

For a non-zero scalar $\lambda$, then $\lambda \textbf{x}$ is parallel to $\textbf{x}$, but may have a different magnitude or direction.

\subsection{Representations of vectors}
\subsubsection{Column form}
Vectors can be represented in column form eg. $\begin{pmatrix}x \\ y\end{pmatrix}$. This means to move $x$ units right and $y$ units up, so $\lambda \begin{pmatrix}x \\y\end{pmatrix}=\begin{pmatrix}\lambda x \\ \lambda y\end{pmatrix}$.

\subsubsection{The unit vectors $\textbf{i}$ and $\textbf{j}$}
The vector $\textbf{i}$ is a unit vector to the right, and the vector $\textbf{j}$ is a unit vector upwards. This means that all vectors can be written in terms of $\textbf{i}$ and $\textbf{j}$:
\begin{equation}
	\begin{pmatrix}x \\ y\end{pmatrix}=x\textbf{i}+y\textbf{j}
\end{equation}

\subsubsection{Magnitude-direction form}
Vectors can also be expressed with a direction (as an angle from an axis), and a magnitude.

\subsection{Magnitudes}
The magnitude of vector $\textbf{x}$ can be represented as $|\textbf{x}|$, and can be found using pythagoras' theorem: if $\textbf{x}=a\textbf{i}+b\textbf{j}$, then $|\textbf{x}|=\sqrt{a^2+b^2}$.

\subsubsection{Unit vectors in a specific direction}
The unit vector in the direction of $\textbf{x}$ can be found by dividing both of the components by the magnitude.

\subsection{Position vectors}
Position vectors are vectors which tell us the position of a point relative to a fixed origin. This means that to vector between two points can be found by using their position vectors (by taking away the first point's position vector and adding on the second's).

\subsection{Modelling with vectors}
We can use vectors to model real-life situations. Velocity, displacement, and force are all examples of vectors (with speed and distance as the magnitudes of the first two). Often these models will involve vectors in magnitude-direction form, perhaps with bearings involved.
