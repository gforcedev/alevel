\section{Modelling with mechanics}
Mechanics can be used to model real world situations. When we want to model a situation, we first consider the real world problem, then we set up our model (which involves deciding what we're going to assume and what variables we're going to consider). Next, we use the model to solve the problem, before interpreting the solution and checking whether the answer is reasonable. If it is, we report the findings in context, otherwise, we may need to check our working or revise our model.

\subsection{Assumptions when modelling}
These are the different components of models for A-level mechanics, and their related assumptions:
\subsubsection{Particles}
Particles have negligible dimensions, with mass concentrated at a single point. Air resistance and rotational forces can be ignored.
\subsubsection{Rods}
Rods have all dimensions except one as negligible. The mass is concentrated along a line, and the object is rigid.
\subsubsection{Laminae}
Laminae have area, but no thickness. The mass is distributed evenly along a flat surface.
\subsubsection{Uniform bodies}
Uniform bodies have evenly distributed mass. We can consider the mass as being concentrated at a single point at the geometrical centre.
\subsubsection{Light objects}
Light objects have a mass which is small compared to other objects. Examples would be a string or a pulley. We treat light objects as having no mass. For strings, tension is the same at both ends.
\subsubsection{Inextensible strings}
Inextensible strings don't stretch under load. This means we assume acceleration is the same in objects connected by strings.
\subsubsection{Smooth surfaces}
Smooth surfaces have no friction between them and objects touching them.
\subsubsection{Rouch surfaces}
If a surface is not smooth, then it is assumed to be rough. Objects in contact with rough surfaces experience a frictional force if they're moving or acted on by a force.
\subsubsection{Wires}
Wires are rigid thin lengths of metal. They are treated as one-dimensional.
\subsubsection{Smooth and light pulleys}
All pulleys are considered to be smooth and light. This means that they have no mass, and the tension is the same on either side.
\subsubsection{Beads}
Beads are particles, with a hole for threading onto a string. They move freely along wires or strings, and tension is the same on either side.
\subsubsection{Pegs}
Pegs are supports from which bodies can be suspended or rested. They are dimensionless, and fixed in space. They can be smooth or rough as specified.
\subsubsection{Air resistance}
Air resistance is a frictional force experienced by objects as they move through air.
\subsubsection{Gravity}
Gravity is the force of attraction between objects. Acceleration due to gravity is denoted by $g$, which is normally approximated to $9.8ms^{-2}$ (unless otherwise specified). We assume that all objects with mass are attracted towards the Earth, and that Earth's gravity is uniform acting vertically downwards.

\subsection{Units}
These are the units used for A-level mechanics.
\begin{table}[ht]
	\begin{tabular}{lll}
		Quantity                 & SI unit                 & Symbol                    \\
		Mass                     & Kilogrammes             & kg                        \\
		Distance or displacement & Metres                  & m                         \\
		Time                     & Seconds                 & s                         \\
		Speed or velocity        & Metres per second       & $ms^{-1}$                 \\
		Acceleration             & Metres per second$^2$   & $ms^{-2}$                 \\
		Force or weight          & Newtons                 & N ($kgms^{-2}$) \\
		Density                  & Kilograms per metre$^3$ & $kgm^3$                  
	\end{tabular}
\end{table}

