\section{Exponentials}
\subsection{Drawing exponential graphs}
Drawing various exponential graphs is fairly easy with knowledge of their basic shape. Usually, it can be done by thinking of more complicated exponential functions as transformations of simpler ones.
\begin{figure}[ht]
    \centering
    \incfig{basic-shape-of-an-exponential-graph}
    \caption{basic shape of an exponential graph}
    \label{fig:basic-shape-of-an-exponential-graph}
\end{figure}

\subsection{$e$}
Euler's constant, $e$, about 2.718 is important for exponentials. For most exponential graphs, their derivative is similar in shape but slightly altered. For $y=e^x$, then the derivative is exactly identical to the original function. In addition, if $f(x)=e^{kx}$, the $f'(x)=ke^{kx}$.

\subsection{Exponential modelling}
Exponentials can be used to model certain real-life values such as population. In populations, then a common model is that the change in population is directly proportional to the current population. This would mean that it is a multiple of the current population. We can use the property that $y=e^{kx}$ differentiates to a multiple of itself to mirror this behaviour in a mathematical model.

\section{Logarithms}
Logarithms can be thought of as another way of writing indices. Instead of writing that $10^2=100$, we can write that $\log_{10}100=2$.

Perhaps a better way of thinking about this would be that writing $\log_{a}b$ is like asking ``what power do I need to raise $a$ to to get $b$?''

\subsection{Notation and labelling}
The subscript number next to the log is called the base. If no base is written, then a base of 10 is implied. $\ln$ means log to base $e$.

\subsection{Some important log equivalencies}
\begin{equation}
	\log_{a}a \equiv 1
\end{equation}
This is because any number to the power 1 is the same number.

\begin{equation}
	\log_{a}1 \equiv 0
\end{equation}
This is because any number to the power 0 is 1.

\subsection{Laws of logarithms}
N.B. These names are made up by Mr Smith rather than standardized. Don't cite ``Because of log rule 1'' or similar when explaining methods.
\subsubsection{Log rule 1}
\begin{equation}
	\log_{a}xy=\log_{a}x+\log_{a}y
\end{equation}
If you have a logarithm of two terms multiplied together, you can split this into logs of the same bases of the two terms added. This also works in reverse - an addition of two logs with the same base can be combined into one log of both of those terms multiplied.

\subsubsection{Log rule 2}
\begin{equation}
	\log_{a}\frac{x}{y}=\log_{a}x-\log_{a}y
\end{equation}
This is extremely similar to the first log rule, but with subtraction corresponding to division (or, if you like, adding negative numbers corresponding to multiplication by their reciprocals).

\subsubsection{Log rule 3}
\begin{equation}
	\log_{a}x^k=k\log_{a}x
\end{equation}
If the term of which the log is being taken has an index, it can be moved to the front of the log, and vice-versa. Be careful with this: $log_{a}x^k$ is NOT the same as $(log_{a}x)^k$, and only the first form works with this rule.

\subsection{Natural logarithms}
Ln, or $\log_{e}$, is known as the natural logarithm. It has the interesting property that its graph is a reflection of $y=e^x$ in the line $y=x$.

You can use the fact that logarithms are the inverses of exponential functions to solve equations based on $e^x$ using natural logs. This can be useful when working with exponential models. Sometimes these equations may be so-called ``quadratics in disguise''. In this case, you can solve as a quadratic for $e^x$ and then work out values for your answers using natural logarithms.

\subsection{Using logs with non-linear data}
We can use logarithms to transform non-linear data into displaying as linear data for the purposes of modelling or displaying. This can be done with polynomial relationships, or exponential relationships.

\subsubsection{Polynomial relationships}
A polynomial relationship between $x$ and $y$ would be $y=ax^n$. By taking logs of both sides, and then using the addition and power log rules, we can transform this relationship into something that looks similar to $y=mx+c$:
\begin{IEEEeqnarray}{rCl}
	y & = & ax^n
	\nonumber\\
	\log y & = & \log ax^n
	\nonumber\\
	\log y & = & \log a + \log x^n
	\nonumber\\
	\log y & = & \log a + n\log x
	\nonumber\\
	\log y & = & n\log x + \log a
\end{IEEEeqnarray}
So if we change $y$ and $x$ for $\log y$ and $\log x$ respectively, we end up with a linear relationship. This means that if we want to linearly represent polynomial relationships on a graph, we can plot $\log y$ against $\log x$.

\subsubsection{Exponential relationships}
An exponential relationship between $x$ and $y$ would be $y=ab^x$. Similarly to before, we can take logs of both sides and apply log rules to end up with something mirroring the equation of a line.
\begin{IEEEeqnarray}{rCl}
	y & = & ab^x
	\nonumber\\
	\log y & = & \log ab^x
	\nonumber\\
	\log y & = & \log b^x + \log a
	\nonumber\\
	\log y & = & x\log b + \log a
\end{IEEEeqnarray}
So if we change $y$ for $\log y$, then we get a linear relationship. This means that if we want to linearly represent exponential relationships on a graph, we can plot $\log y$ agaist $x$.
