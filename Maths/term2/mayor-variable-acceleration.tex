\section{Variable acceleration}
When acceleration is variable, we can express acceleration, velocity, and displacement of an object as arbitrary functions of $t$. This means that we can do things like finding minimum/maximum velocities and when then occur, as this is just doing pure maths things to the equations / graphs.

\subsection{Using calculus to express suvat variables}
Velocity is the change in displacement over time:
\begin{equation}
	v=\frac{ds}{dt}
\end{equation}
Acceleration is the change in velocity over time:
\begin{equation}
	a=\frac{dv}{dt}=\frac{d^2s}{dt^2}
\end{equation}
This means that we can differentiate from displacement to find velocity and / or acceleration.

We can also go the other way and use integration starting from either velocity or acceleration to find functions to express the other variables. If this is necessary for a question an initial state will be given so that the constant of acceleration can be worked out.
\begin{equation}
	v=\int (a) dt
\end{equation}
\begin{equation}
	s=\int (v) dt = \iint (a) dt
\end{equation}

\subsection{Using calculus to prove suvat equations}
\subsubsection{Proof for $v=u+at$}
\begin{IEEEeqnarray}{rCl}
	v & = & \int (a) dt
	\nonumber\\
	v & = & at + c
	\nonumber
\end{IEEEeqnarray}
When $v=u$, $t=0$, so $c=u$. This implies that:
\begin{equation}
	v=u+at
\end{equation}

\subsubsection{Proving $s=ut+\frac{1}{2}at^2$}
\begin{IEEEeqnarray}{rCl}
	s & = & \int(v) dt
	\nonumber\\
	s & = & \int(u+at) dt
	\nonumber\\
	s & = & ut+\frac{1}{2}at^2+c
	\nonumber
\end{IEEEeqnarray}
When $t=0$, $s=0$, so $c=0$. This implies that:
\begin{equation}
	s=ut+\frac{1}{2}at^2
\end{equation}
