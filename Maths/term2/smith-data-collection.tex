\section{Data collection}
\subsection{Terminology for stats}
A population is the set of all the things that we're interested in gathering data about.

A sample measures a subset of the population. We take samples so we can analyse them and infer something about the whole population.

A census measures every member of the population.

The advantages of a census over a sample is that it gives a more accurate view of the population, whereas a sample could be misleading. However, censuses are impossible if the population size is infinite, and also impractical if the measurement damages the member being measured (for example, testing to see how many times a hinge can be opened and closed before it fails).

A parameter is a number describing an entire population.

A statistic is a number taken from a single sample.

A sampling unit is an individual member of the population.

A sampling frame is a list of sampling units.

\subsection{Random sampling methods}
\subsubsection{Simple random sampling}
This sampling method is when a set of random numbers are generated, either with a computer or by picking them out of a hat or some such method (known as lottery sampling).

This sampling method is free from bias, and fairly easy to implement for small populations and small samples. Each sampling unit is equally likely to be selected. However, it's not suitable when either the population size or required sample size are extremely large, and it does come with the necessity to choose a sampling frame (which a census would not).

\subsubsection{Systematic sampling}
In systematic sampling, the sampling units are chosen at a set interval from an ordered list. For example, people might be ordered by height and then every 10th person chosen. The first person must be chosen at random (so here pick a random number between 1 and 10).

Systematic sampling is fairly simple to use, and works well with large samples or population sizes. However, if a bad sorting method is chosen it could introduce bias, and again it needs a sampling frame unlike a census.

\subsubsection{Stratified sampling}
In stratified sampling, the population is divided into mutually exclusive strata, and then a number of sampling units from each stratum, proportional to the stratum's share of the whole population, are randomly chosen.

Stratified sampling is good because it accurately reflects the population's structure, and guarantees that groups will be proportionally represented in the sampling frame. However, bias can be introduced if strata are badly chosen, and it is more complicated to perform. Another thing to note is that selection within each strata has the same advantages and disadvantages as systematic sampling.

\subsection{Non-random sampling methods}
\subsubsection{Opportunity sampling}
Opportunity sampling is when an interviewer or researcher simply samples whatever is easiest. This might be going out onto the street to ask people their opinions (perhaps people who match a specific criteria), or simply testing the first few products that come off a production line on a particular day.

Opportunity sampling has the advantage of being easy to carry out. However, it's unlikely to be representative of the whole population, and very dependent on the specific research methods used (for example, many psychological studies may be inaccurate for the entire world population because they are mostly done on western undergraduates).

\subsubsection{Quota sampling}
Quota sampling is similar to stratified sampling, crossed with opportunity sampling. Quotas of sampling units with particular characteristics are determined beforehand, and then sampling units are tested for the data being collected, and also which quota they fit into. The researcher then keeps testing until all of the quotas have been filled.

Quota sampling allows reserachers to get a representative sample of the population while remaining farly easy to carry out. However, it can introduce bias for similar reasons to opportunity sampling, and also has the disadvantages of stratified sampling concerning splitting populations into inappropriate groups.

\subsection{Data}
Data can be split into primary and secondary data. This is where primary data was collected specifically to answer your question, while secondary data was collected for another reason and happens to be useful for answering your question as well.

Non-numerical data is called qualitative data. Examples would be colours or types of material.

Numerical data is called quantitative data. It can be split into continuous data which could be measured to any degree of accuracy, or discrete data which can only take certain values (for example, counting something could only ever yield a whole-number result).

Continuous data is generally collated into ranges, for example $0 \leq h < 10$ and $10 \leq h < 20$, while discrete data is generally collated into groupings such as 0 to 10 and 11 to 20.
