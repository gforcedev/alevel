\section{Forces and motion}
\subsection{Newton's first law}
Newton's first law states that an object's velocity will remain constant unless it's acted on by an unbalanced force.

This means that the resultant force, or sum of all the forces acting on the object, is not equal to 0.

This means that for objects in equilibrium (the resultant force is 0), the velocity will stay the same.
\subsection{Newton's second law}
Newton's second law states that teh resultant force acting on an object is equal to the product of its mass and its acceleration:
\begin{equation}
	F=ma
\end{equation}

\subsection{Connected particles}
Modelling with connected particles requires deciding whether to consider them as one object using Newton's second law on the entire system, or whether to consider each connected particle separately. Many questions will require both, perhaps working out the resultant force on the system in order to work out a frictional force on one particle, and then considering the objects separately in order to work out tension.

In order to consider objects as a single particle, they must remain in contact, or be connected by an inextensible rod or string.

\subsection{Newton's third law}
Newtons third law says that for every action, there is an equal and opposite reaction. This is useful when modelling because we can find forces exerted by objects by working out and reversing the forces exerted on them, or vice-versa.

\subsection{Pulleys}
Some common questions with pulleys might be to write the equation of motion. This means applying $F=ma$ to the particles in the system.

Another common pulley-related question might be to say how the solution used the assumption that the pulley is frictionless. In this case the answer is assuming tension was the same on both sides of the pulley.
